\documentclass[letterpaper,10pt]{article}
\usepackage{nuvem}
\usepackage[usenames,dvipsnames]{xcolor}
\usepackage{hyperref}
\hypersetup{
	%pagebackref=true,
	pdfcreator={LaTeX with abnTeX2},
	pdfkeywords={abnt}{latex}{abntex}{USPSC}{trabalho acadêmico}, 
	colorlinks=true,       		% false: boxed links; true: colored links
	linkcolor=blue,          	% color of internal links
	citecolor=blue,        		% color of links to bibliography
	filecolor=magenta,      		% color of file links
	urlcolor=blue,
	allbordercolors=black,
	bookmarksdepth=4
}
\usepackage[utf8]{inputenc}


\begin{document}

\title{Analytic consideration of the society by itself}

\author{Marilia M. Pisani$^1$\\
Renato Fabbri$^2$}
\address{$^1$CCNH/UFABC, $^2$IFSC/USP}
\email{$^1$marilia.m.pisani@gmail.com, $^2$renato.fabbri@gmail.com}

\begin{abstract}
Social participation has been growing on the last decade despite
the recent rise of conservative political parties 
in various locations of the world.
We aim to consider a set of developments on linked social data
and analysis performed over the last years, with a focus
on the potential use for the civil society and scientific academy.
Various conceptualizations have been gathered by interviewing specialists
and State authorities and have been validated by them and their institutions.
Data from known social networks, such as Facebook, Twitter, IRC and Email,
and from more specialized social participation platforms, have been
	translated to RDF and linked to these conceptualizations in the form of OWL ontologies~\cite{losd,pnud5}.
Social networks were reported as very stable and their language varies
	with connectivity~\cite{tese,stab}.
	Also, resources recommendation and experiments have been performed with such data.
	How are we to articulate the gathering and analysis 
	of such data with the needs of the civil society and the academy?
	Will it be valuable to deal with the private and State interests
	that shape our society?
	Does Anthropological Physics yield reasonable strategies
	to collect and analyze data from our social structures?
	How does the global semantic web of linked data~\cite{losd}
	relate to linked social data?
\end{abstract}

\section{Introduction}
Social analytics is a term associated to Lars-Hendrik Schmidt~\cite{wSA}
in philosophy.
One might also relate social analytics to social data mining
by extension of the terms: visual analytics, data analytics, social media analytics etc,
being used nowadays (e.g. look for these terms in the English language Wikipedia).
Interestingly, both social analytics 
have among core practices the reporting of tendencies of the times.
And they put emphasis on issues and analytical outcomes
that are very controversial and result from recent technological and scientific developments.

On the critical theory tradition,
Marcuse have a central role on
questioning the conditions
imposed for living by technological advances~\cite{m1,m2,m3,m4}.
And the consideration of the authoritarian personality traces~\cite{au}
has gained relevance due to unforeseen social phenomena
such as the recent and worldwide rise of the right-winged manifestations
and political representation~\cite{rise,rise2}.

On physics and computer sciences,
both the advent of the complex networks
theory and the big data practices
had an impact on the utility of considering
social systems.
One enables forecast of properties
and the use of natural laws for deriving understandings.
The other provides the data that represents the real systems
in scales that make them useful for management and
scientific studies (including experiments as in Sections~\ref{exp} and~\ref{self}).
Social networking platforms are massively and constantly used and thus fed.
They are most often elaborate suites of software with
an interface for the user to interact with other users
and often makes use of these recent scientific and technological
developments.\cite{tese}

In this context,
we advocate the analytic consideration
of the Society by itself.
There are many ways to achieve this,
and it is performed to some extent~\cite{scriptLattes,losd,p5,an,an2}.
Our focus here is on the mining of the data
derived from social platforms.
It is not very trivial in our context
of monetary obsession to achieve engagement
in such analytical undertake.
Even if there is money making in the process,
the citizens already have their livelihood activities.
Therefore, some effort has been given to
enable this kind of `social participation'.
Core goals are:
\begin{itemize}
	\item Develop the critical view, on specialists and layman,
		of social media and
		management performed by the State and the private sector.
	\item Make advances in the data available for consideration
		and the algorithms for analysis.
	\item Implement or study the possibility of implementation
		of software systems.
\end{itemize}

\section{What has been done}
On the following sections, we will focus on
developments related to social systems and technology.
There are methods, software and scientific discoveries
that underpins them~\cite{stab,tese,kolm}.

\subsection{Social participation data}\label{losd}
Data from Facebook, Twitter, Email lists, IRC Channels, ParticipaBR,
Cidade Democrática and AA have been translated into linked data (RDF)
and has been named the LOSD (Linked Open Social Data~\cite{losd}).
Initial OWL ontologies were derived from the data by automated routines.
The dataset is oriented towards general scientific research
and the development of analytic interfaces~\cite{p1,p2,p3,p4,p5}.
Even so, it has been used e.g. by participants of a SESC workshop
and has received some attention from the scientific community~\cite{drones,tese},
and mining of the data revealed very stable patterns as predicted by natural laws
and strong differentiation of the language of participants across hubs,
intermediary and the periphery~\cite{stab,tese}. 

\subsection{Ontologies}\label{ont}
Beyond the ontologies mentioned in the last section,
there are social participation ontologies that describe
social participation instances and mechanisms~\cite{p5}.
Such instances and mechanisms are:
Conference (Conferência), Forum (Fórum),
Committee (Comitê), Council (Conselho), Ombudsman (Ouvidoria),
Public Consultation (Consulta Pública), Dialog Table (Mesa de diálogo),
Monitoring Table (Mesa de Monitoramento), Intercouncil Forum (Fóruns Interconselhos),
Audience (Audiëncia),
Virtual Environment (Ambiênte Virtual).
They received additional ontologies relating them to documents and
larger scopes.

These ontologies are contextualized by the whole semantic web
(or Giant Global Graph~\cite{losd})
and by other social participation ontologies:
the OPa (Ontology of ParticipaBR~\cite{p5}), the OPS (Ontology of Social Participation~\cite{ops}),
the OCD (Ontology of Cidade Democrática~\cite{p5}), the OAA (Ontology of AA~\cite{p5}).
There are minor ontologies, such as the one about the Magic Box (Caixa Mágica~\cite{cm},
a social participation wifi gadget).
These OWL ontologies (and the SKOS vocabularies which most of them have, following
a pattern suggested in~\cite{ont}) enable
sound conceptual discussions, navigation by humans and machine (data discovery),
 automated inference; and data organization, integration and linkage by means of conceptualizations.
It seems reasonable to use these ontologies to link the social participation data
and enable a scalable participatory legacy of machine- and human- friendly representation
of data.

\subsection{Critical theory and anthropological physics}\label{exp}
The potential for prejudice and the study of the self
were considered in~\cite{imp,stab,tese}.
In summary, the hubs, intermediary and periphery sectorialization
of a network might be achieved (or imposed) in a social network.
It yields a classification of the participants
by the comparison of the degree (or strength) distribution
against that of an Erdös-Rényi network.
Hubs are valued by literature
(although the intermediary are also reported as authorities
and structuraly most important agents)~\cite{ega},
which favors the potential for prejudice
of a quantitative classification (or typology)
of human beings.
At the same time, the classification of the participants
(hub, intermediary or peripheral)
vary in time constantly~\cite{stab,tese}
and across scales and networks,
which minimized the potential
for prejudice of such classification.

Violence, prejudice and authoritarianism,
and the authoritarian personality and technology,
are frequently co-occurring themes in the two dossier of the
Nexos Networks for Critical Theory and Interdisciplinary Research~\cite{nexos1}.
We also considered the cold-blooded reception of the reality
enabled technological devices such as drones~\cite{drones}.

The Anthropological Physics is concerned
with the observation of natural
phenomena in the social structures
of the observer and has been
considered in~\cite{an,an2,tese}.
Comte talked about anthropological physics,
but it had the sense which physical (or biological or natural)
anthropology has, which is a branch of anthropology
(the other is social anthropology).
Anthropological physics, in the sense we use and understand correct,
is physics (research on the natural phenomena mainly of complex networks)
with anthropological aspects.
The writing and study of diaries is a traditional ethnographic asset,
which resembles the collection and study of one's data by the person itself,
social data.
The current guidelines are centered on keeping the process open,
using free and open source tools and file formats,
and publicly accessible repositories of software, gadgets, data, writings, etc;
while gathering and studying the social data related to the researcher
(and other data if needed and afterwards).~\cite{an,an2,imp,tese}



\subsection{Self-transparency, AA and the fundamental cycle}\label{self}
A system for sharing work, development and research processes
(actually, any dedication)
is described in~\cite{aa,aa2}.
It is the Algorithmic Autoregulation (AA) methodology and it has
received a number of software implementations.
It is based on voluntary logging of messages (called shouts)
about what is being done (reading, coding, etc).
The Shouts may
be linked to sessions and screencasts or be blind-reviewed
by other users.
The methodology and software have initiated discussions
e.g. about documenting academic dedication not only though
traditional documents (e.g. articles, books) and participation (e.g. conferences, thesis defense),
as currently performed e.g. by Lattes Curriculum.
Such documentation (log of dedications)
would enable one to dedicate more time to individual
tasks and make more paced and relevant contributions.

The concept has reached other initiatives, such
as the Brazilian federal portal of social participation
ParticipaBR.
And have incited discussions that yield interesting concepts,
such as the fundamental cycle,
which is an idealized interaction between
Society and a management sector of it,
such as the State.
The individuals that feel the need to
change the way living is imposed by the State
should be able to act on the State,
have results and get paid for the time they put
into performing the social participation.
The existence of such a cycle would erradicate
extreme poverty, as the poor would be paid
to act and change State measures for the poor
or a community.
Thus, the ideal fundamental cycle is a social
participation model which is very shabbily practiced
and is defined by giving the one in need good conditions
to change the context which puts he/she in need.
The mechanisms or institutions that give the
individual the conditions and means to act
is then called the State or Government or,
more accurately nowadays, the Institutionalized Power.
Notice that, as such, the institutionalized power
have the goal to allow the interference 
of the individual in its management
and that this is extended to the private initiatives.
Social participation in the private sector
seems to be very incipient.



\subsection{Audiovisual Analytics platform}
Described in~\cite{nuvem2}, it is an envisioned audiovisual analytics
software for social data, with focus on network and textual data.
It will also be sent to the I Workshop @NUVEM, together with
this document, for separate appreciation.

\section{What shall we do?}
Most certainly we will
make the LOSD and social participation ontologies (Sections~\ref{losd} and~\ref{ont})
available e.g. on DataHub and Data.World.
In the Nexos research network, we
considered a number of times to study authoritarian personality traces~\cite{au}
on this data.
And to further consider the Anthropological Physics aspects that
arise from studying our own social structures~\cite{an,an2}.
The self-transparency is having some attention in scientific writings
and hacks~\cite{aa,aa2} but seems to be lacking user bases.
The developments for Audiovisual Analytics are incipient
and have the development focus of only one researcher and developer.
The Nexos research network is very active nation wide.
It has an overall tendency to tackle subjects within the
critical theory tradition and thus considers the society
and potential enhancements through criticism of its organization,
which is valuable for the technological developments being made~\cite{nexos1}.

Research groups @Nuvem of Cloud Computing and Intelligent Societies might have 
further context to share about these issues.
For example, are there SparQL endpoints with (Brazilian) social participation
(including self-transparency) data?
Is it necessary to contact lawyers to
better know the limits of our possibilities
to gather and research our social data within
the Anthropological Physics perspective
and are there well known guidelines?
Is there more participatory linked data available
in Brazil? Ontologies?
We might benefit from directions 
on better linking LOSD to the semantic web
(DBPedia, other participatory data, etc),
and for a reasonable way to keep the data online
(through DataHub or Data.World or both?
A Pubby-like interface?),
and to develop an audiovisual analytics
software
(persistence, web?, analysis methods, audiovisual rendering, etc,
as described in~\cite{nuvem2}).

Other questions that might initiate
nice discussions or entail collaboration are:
how to enable a self-transparency user base?
Where to keep the linked data?
How to manage the ontologies and keep their
constant development (as needed and predicted in the literature)?
How to achieve a reasonable 
use of our social (open linked) data?
Is it possible to have an audiovisual analytics
platform with our own (scraped) social data,
with facilities for media rendering and
interaction experiments (collection and diffusion of information),
that enables the user base and interested parties to
collective gather social data, analyses and conceptualizations?
Is the societal consideration of our social data
relevant for equilibrium with the State and private sectors?
Is it possible to achieve social participation in the private sector,
e.g. to have civil representation in the management of companies,
to regulate matters such as YouTube adds in Google,
nutrient standards in McDonald's,
and other things in other major and minor institutions?

\begin{thebibliography}{99}

\bibitem{losd}
	Fabbri, R., \& de Oliveira, O. N. (2016). Linked Open Social Database. Github repositories, from \url{https://github.com/ttm/linkedOpenSocialDatabase/raw/master/paper.pdf}

\bibitem{pnud5}
Fabbri, R. (2014). Social participation ontologies and rules to fuel a social participation linked data cloud.
	Technical report for the United Nations Development Program.
		From \url{https://github.com/ttm/pnud5/raw/master/latex/produto.pdf}

\bibitem{tese}
Fabbri, R. (2017). Topological stability and textual differentiation in human interaction networks:
		statistical analysis, visualization and linked data. Doctoral thesis.
		From \url{https://github.com/ttm/thesis/raw/master/thesis-rfabbri.pdf}

\bibitem{stab}
Fabbri, R., Fabbri, R., Antunes, D. C., Pisani, M. M., \& de Oliveira, O. N. (2017). Temporal stability in human interaction networks. Physica A: Statistical Mechanics and its Applications.

\bibitem{wSA}
	Social analytics. (2017, February 5). In Wikipedia, The Free Encyclopedia. Retrieved 11:37, November 12, 2017, from \url{https://en.wikipedia.org/w/index.php?title=Social\_analytics\&oldid=763744127}

\bibitem{m1}
	Marcuse, H. (2013). One-dimensional man: Studies in the ideology of advanced industrial society. Routledge.

\bibitem{m2}
	Marcuse, H. (2009). A responsabilidade da ciência. Scientiae studia, 7(1), 159-164.

\bibitem{m3}
	Marcuse, H. (2004). Technology, War and Fascism: Collected Papers of Herbert Marcuse (Vol. 1). Routledge.

\bibitem{m4}
	Pisani, M. M. (2009). Algumas considerações sobre ciência e política no pensamento de Herbert Marcuse. Scientiae Studia, 7(1), 135-158.
\bibitem{au}
	Adorno, T. W., Frenkel-Brunswik, E., Levinson, D. J., \& Sanford, R. N. (1950). The authoritarian personality.

\bibitem{rise}
Kaufmann, E. (2017). Interview with Eric Kaufmann: cultural values and the rise of right-wing populism in the West. British Politics and Policy at LSE.

\bibitem{rise2}
	Gaarsted, J. B., \& Agustín, L. R. (2017). The rise of right-wing populism.

\bibitem{scriptLattes}
	Mena-Chalco, J. P., \& Júnior, C. E. S. A. R. (2013). Prospecção de dados acadêmicos de currículos Lattes através de Scriptlattes. Bibliometria e Cientometria: reflexões teóricas e interfaces. São Carlos: Pedro \& João.

\bibitem{p5}
Fabbri, R. Social participation ontologies and rules to fuel a social participation linked data cloud, 2014.
	Technical report for the United Nations Development Program, made with the Brazilian Presidency.
		Available at \url{https://github.com/ttm/pnud5/raw/master/latex/produto.pdf}

\bibitem{an}
	Fabbri, R. What are you and I? [anthropological physics fundamentals], 2015. Available at \url{https://www.academia.edu/10356773/What\_are\_you\_and\_I\_anthropological\_physics\_fundamentals\_}

\bibitem{an2}
	Anthropological physics and social psychology in the critical research of networks. Complex Networks Digital Campus (CS-DC'15).
	Em \url{http://cs-dc-15.org/papers/cognition/social-psychology/anthropological-physics-and-social-psychology-in-the-critical-research-of-networks/}

\bibitem{kolm} 
  R.~Fabbri and F.~G.~De León,
  ``A Statistical Distance Derived From The Kolmogorov-Smirnov Test: specification, reference measures (benchmarks) and example uses,''
  Anais do XX ENMC - Encontro Nacional de Modelagem Computacional e
  VIII ECTM - Encontro de Ci\^encias e Tecnologia de Materiais, Nova Friburgo,
  RJ - 16 a 19 Outubro 2017
		[arXiv:1711.00761 [physics.data-an]]. Available at: \url{https://arxiv.org/abs/1711.00761}.

\bibitem{p1}
	Fabbri, R. Primeira ontologia do Portal Federal de Participação Social Descrição e código OWL: Ontologia do ParticipaBR, 2014.
    United Nations Development Programme and Brazilian Presidency of the Republic. Available at \url{https://sourceforge.net/p/labmacambira/fimDoMundo/ci/master/tree/textos/ontologia/ontologiaParticipa\_.pdf}

\bibitem{p2}
    Fabbri, R. United Nations Development Programme: ParticipaBR context, triplification of data and example of usage, 2014.
    Available at \url{http://sourceforge.net/p/labmacambira/fimDoMundo/ci/master/tree/textos/SparQL/triplificaDisponibiliza.pdf?format=raw}

\bibitem{p3}
    Fabbri, R. United Nations Development Programme: Tools for content classification in the ParticipaBR Brazilian federal portal of social participation, 2014.
    Available at \url{https://github.com/ttm/pnud3/blob/master/latex/produto.pdf?raw=true}

\bibitem{p4}
    Fabbri, R. United Nations Development Programme: Adaptations and increments for the ParticipaBR Brazilian federal portal of social participation, 2014.
    Available at \url{https://github.com/ttm/pnud4/blob/master/latex/produto.pdf?raw=true}

\bibitem{drones}
	Pisani. M.
	Entre Drones e Ciborgues: algumas considerações sobre teoria crítica e a necroética das redes sociais. Revista Impulso, Nov/2017.

\bibitem{ops}
	Fabbri, R., de Luna, R. B., Martins, R. A. P., Amanqui, F. K. M., Moreira, D. D. A., Oliveira Jr, O. N. O. (2015). Social Participation Ontology: community documentation, enhancements and use examples. arXiv preprint arXiv:1501.02662. Available at \url{https://github.com/ttm/ops/raw/master/plosOne/plosComImagens.pdf}

\bibitem{cm}
	Fabbri, R. Magic Box Ontology.
		Available at \url{https://github.com/ttm/pnud5/raw/master/latex/produto.pdf}

\bibitem{ont} 
  R.~Fabbri and M. Pisani,
  ``Enhancements of linked data expressiveness for ontologies,''
  Anais do XX ENMC - Encontro Nacional de Modelagem Computacional e
  VIII ECTM - Encontro de Ci\^encias e Tecnologia de Materiais, Nova Friburgo,
  RJ - 16 a 19 Outubro 2017
		Available at: \url{https://github.com/ttm/equanimousScalefree/raw/master/article/article.pdf}.

\bibitem{imp}
	Fabbri, R. Redes complexas para redes sociais: introdução, aspectos críticos e software, 2017.
		Revista Impulso,
		ISSN 2236-9767,
		DOI 10.15600/2236-9767/impulso.v27n69p65-75.
		Disponível em
		\url{https://www.metodista.br/revistas/revistas-unimep/index.php/impulso/article/view/3367}.

\bibitem{ega} 
  R.~Fabbri and M. Pisani,
  ``Egalitarian aspects of scale-free networks,''
  Anais do XX ENMC - Encontro Nacional de Modelagem Computacional e
  VIII ECTM - Encontro de Ci\^encias e Tecnologia de Materiais, Nova Friburgo,
  RJ - 16 a 19 Outubro 2017
		Available at: \url{https://github.com/ttm/equanimousScalefree/raw/master/article/article.pdf}.

\bibitem{nexos1}
	Dossies of the Nexos Network for Critical Theory and Interdisciplinary Research. Available at: \url{https://www.metodista.br/revistas/revistas-unimep/index.php/impulso/issue/view/207} and \url{https://www.metodista.br/revistas/revistas-unimep/index.php/impulso/issue/view/207}

\bibitem{aa}
	Fabbri, R., Fabbri, R., Vieira, V., Penalva, D., Shiga, D., Mendonça, M., Negrao, A., Zambianchi, L., \& Thumé, G. (2013). AA: The Algorithmic Autoregulation (Distributed Software Development) Methodology. RESI. From \url{https://arxiv.org/abs/1604.08255}

\bibitem{aa2} 
  R.~Fabbri,
  ``The Algorithmic-Autoregulation (AA) Methodology and Software: a collective focus on self-transparency,''
  Anais do XX ENMC - Encontro Nacional de Modelagem Computacional e
  VIII ECTM - Encontro de Ci\^encias e Tecnologia de Materiais, Nova Friburgo,
  RJ - 16 a 19 Outubro 2017
		Available at: \url{https://github.com/ttm/ensaaio/raw/master/emc/article.pdf}.

\bibitem{nuvem2} 
  R.~Fabbri and M. Pisani
  ``Audiovisual Analytics of social data,''
  I Workshop @NUVEM, UFABC campus de Santo André,
		Available at: \url{https://github.com/ttm/aavo/raw/master/latex/nuvem/nuvem2.pdf}.
\end{thebibliography}


\end{document}
