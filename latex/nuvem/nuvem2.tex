\documentclass[letterpaper,10pt]{article}
\usepackage{nuvem}


\begin{document}

\title{Audiovisual Analytics of social linked data}

\author{Renato Fabbri$^1$\\
	Marilia M. Pisani$^2$
}
\address{$^1$IFSC/USP, $^2$CCNH/UFABC}
\email{$^1$renato.fabbri@gmail.com, $^2$marilia.m.pisani@gmail.com}

\begin{abstract}
	Audiovisual analytics of social data is useful for State and private
	management, scientific and individual endeavors.
	If you have data as information in a computer, and wants to
	transmit it to an observer or user, the use of visual cues is an obvious,
	and sometimes considered optimized, choice because of the efficiency and
	complexity of our visual cognition.
	Our sonic cognition is sometimes considered of greater complexity,
	due to patterns of music and spoken language.
	How to correctly formalize the mapping of data to physical stimuli
	and then to the perceived stimuli
	respecting the Webner-Fechner and Steven's laws?
	Related to sound, most traditional notation and musical qualities
	respect the Webner-Fechner law, with note grids (e.g. semitones, octaves),
	respecting the logarithm of frequency: iterval$=log_2(f_1/f_0)$,
	loudness$=log_10(p_1/p_0)$
	Duration seems to be conceptualized as such
	because musical notation uses the division in two (simple, imperfectus),
	amd three (perfectus) yielding a grid having
	$d_i=d_{i+1}/2$ and $d_i=d_{i+1}/3$.
	$durations=2^{x-i}$ and $=3^{x-i}$.
	Which implies $log_2(dur) = x-i$.
	A Webner-Fechner compliant relation
	between the number of the durations and
	depth of the division x-i = (4, 8, 16) for simple,
	(3, 9, 12) for compound,
	(5,7,13) complex.
	6 is compound in traditional music theory,
	but is the only one that relates periods of 2 and 3,
	subdivision of 2 and 3.
\end{abstract}

\section{Introduction}
Linked data is useful for data integration from different sources,
for semantic representation and manipulation of data,
such as by automated inferences,
and sound conceptual considerations of both domain and context
knowledge.
The mapping of data to audiovisual cues
involve a number of routines, conceptual frameworks,
and data.
The scaling and integration factors are decisive in setting
of Big Data such as in social data mining.
The study of data that expresses human social systems
poses ethical and conceptual challenges~\cite{an,an2}.
An overview of the Linked Open Social Data gathered
in RDF and related ontologies and vocabularies
can be found in~\cite{nuvem1,tese,losd}.
Some limitations of current linked data standards
are described and potential solutions given in~\cite{ont}.

\section{Current Framework}\label{current}
We have strived to obtain relevant algebraic, empiric, conceptual,
analyses of social structures~\cite{tese,stab},
often supported by software, which is a very formal representation
of what is being performed (although also very incomplete).
In this section, we focus on current efforts around Audiovisual Analytics.

\subsection{AAVO}
AAVO is a first tentative ontology for Audiovisual Analytics.
Figure~\cite{aavo} depicts the AAVO Core,
and both Core and Extended AAVO are well described in~\cite{aavoA}.
The following concepts are envisioned for the core:

And it might receive other layers of conceptualizations,
such as for data type, colors, charts, etc.

Its purposes are: automated inferences and recommendations on
contexts involving data visualization;
enable relevant theoretical discussions by having an
established and formal conceptualization;
provide a representation of data that is friendly for
both humans and machines (in browsing, discovery, inference);
underpin data integration;
provide the conceptual and data architecture for
the audiovisual analytics software described in
the next section.

\subsection{An audiovisual analytics platform of social systems}
A software system, a platform for audiovisual analytics of big data
have been envisioned which have the following characteristics:
\begin{itemize}
  \item Interface has capabilities of making automated and periodic changes of
	  the presentation of the content (data and analyses).
		This entails a fractal and musical consideration
		of the audiovisualzations.
		User might pause and set changing patterns..
  \item The simultaneous use of both the vision and hearing channels.
	  User might mute or focus on the sonic cues.
  \item The system has the purpose of providing an aesthetic experience
	  to the user.
		This includes facilities to render audiovisual media
		for aesthetic appreciation or attractive representation of
		analyses~\cite{tufte}.
  \item Keeps records of state through users and sessions.
	  Each session has a set of states, each state a set
		of interface parametrization and data references.
	Annotations might be linked to users, sessions, states, data,
		analysis methods, widgets, etc; and categorized
		by keywords or simple values.
  \item Persistence (probably through the web browser's sandbox)
	  to allow a user to reload already downloaded software
		and data,
		and to allow a user to work offline.
  \item Sharing of sessions, states, data, and annotations among users,
	  preferably in real time (as achieved through Meteor.js).
		Sharing of media to non-users and for presenting results.
  \item A written interface to control the system which should be
	  suitable for scripting, one line commands, keyboard shortcuts.
		(The written interface expresses a very broad space
		of possible commands.)
  \item Linked data to achieve a formal and browse-friendly representation of data for both humans and machines;
	  for a unified consideration of both data and knowledge architecture;
facilitates the linkage of metadata from users;
facilitates resources recommendations.
  \item Emphasis on the analysis of networks and text (and their simultaneous analysis),
	  reason why it is fit for social networks, but also for general complex networks
		and textual data.
  \item Designed to be enhanced with use by means of tests with users, of analyses of sessions and states
	  (maybe have facilities to record mouse movement, clicks and keyboard strokes).
\end{itemize}

The temporal dynamic of the interface is useful to do a rapid scan of the data by parts
Observing data by parts is a core procedure in mining big data.
Persistence is also valued for the purpose of analyzing big data.
Complex networks and text yield a broad user base
and set of knowledge fields which might be studied withing the system.
The focus on social systems entails analysis results of interest
to both the scientific research, and State and private sectors.
The artistic course of current and envisioned developments
is useful for formal documents but also favors the engagement
of layman.

\subsection{Perceptual framework}
Link to AAVO: the three sampling groups.
\subsection{Other: Erdös Sectioning, KS-derived statistics, Wordnet,
text mining, stability and differentiation, partnerships}
\subsection{What to do?}
We will surely integrate the Hypothesis, Goals, Interaction
concepts into the AAVO Core.
We should develop the system in Section~\ref{sys} to some extent,
maybe for concept proofs or simple prototyping.

With the research groups @Nuvem, 
relate the missing concepts in AAVO,
enhance the software design of the system,
Know about the right tools for visualizing
ontologies and social data.
For knowing about pertinent analysis methods
and known conceptualizations.
Virtual reality, visualization and societal groups
might want to formalize their own conceptualizations,
and concepts related to AAVO might have similar developments
or which are confluent (e.g. symbiotic)
with Section~\ref{current}: software
theoretical framework, research and institutional experiments and equipment,
linked data, and other initiatives or knowledge.








\begin{thebibliography}{99}

\bibitem{krishnan00} E. Krishnan, A. M. Shan, T. Rishi, L. A. Ajith, C. V.
Radhakrishnan, \textit{On-line Tutorial on \LaTeX{}},
``Mathematics'' (Indian \TeX{} Users Group, 2000), \\
\url{http://www.tug.org/tutorials/tugindia/chap11-scr.pdf}.

\bibitem{vantrigt97} C. van Trigt, ``Visual system-response functions and estimating reflectance,''
J. Opt. Soc. Am. A \textbf{14}, 741--755 (1997).

\bibitem{masters93} T. Masters, \emph{Practical Neural Network Recipes in C++} (Academic, 1993).

\bibitem{shoop97} B. L. Shoop, A. H. Sayles, and D. M. Litynski, ``New devices for optoelectronics: smart pixels,''
in \emph{Handbook of Fiber Optic Data Communications},
C. DeCusatis, D. Clement, E. Maass, and R. Lasky, eds. (Academic, 1997), pp. 705--758.

\bibitem{kalman76} R. E. Kalman,``Algebraic aspects of the generalized inverse of a rectangular matrix,'' in
\emph{Proceedings of Advanced Seminar on Generalized Inverse and Applications}, M. Z. Nashed, ed. (Academic, 1976), pp. 111--124.

\bibitem{craig96} R. Craig and B. Gignac, ``High-power 980-nm pump lasers,''
in \emph{Optical Fiber Communication Conference}, Vol. 2 of 1996 OSA Technical Digest Series (Optical Society of America, 1996), paper ThG1.

\bibitem{steup96} D. Steup and J. Weinzierl, ``Resonant THz-meshes,''
presented at the Fourth International Workshop on THz Electronics, Erlangen-Tennenlohe, Germany, 5--6 Sept. 1996.

\end{thebibliography}


\end{document}
