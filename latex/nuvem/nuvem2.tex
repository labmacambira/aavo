\documentclass[letterpaper,10pt]{article}
\usepackage{nuvem}
\usepackage[usenames,dvipsnames]{xcolor}
\usepackage{hyperref}
\hypersetup{
	%pagebackref=true,
	pdfcreator={LaTeX with abnTeX2},
	pdfkeywords={abnt}{latex}{abntex}{USPSC}{trabalho acadêmico}, 
	colorlinks=true,       		% false: boxed links; true: colored links
	linkcolor=blue,          	% color of internal links
	citecolor=blue,        		% color of links to bibliography
	filecolor=magenta,      		% color of file links
	urlcolor=blue,
	allbordercolors=black,
	bookmarksdepth=4
}
\usepackage[utf8]{inputenc}


\begin{document}

\title{Audiovisual Analytics of social linked data}

\author{Renato Fabbri$^1$\\
	Marilia M. Pisani$^2$
}
\address{$^1$IFSC/USP, $^2$CCNH/UFABC}
\email{$^1$renato.fabbri@gmail.com, $^2$marilia.m.pisani@gmail.com}

\begin{abstract}
	Audiovisual analytics of social data is useful for State and private
	management, scientific and individual endeavors.
	If there is data, information or knowledge in a computer,
	and the observer or user should have it, the employment of visual cues is an obvious,
	and sometimes considered optimized, strategy because of the efficiency and
	complexity of our visual cognition.
	Our sonic cognition is sometimes considered of greater complexity,
	due to patterns of music and spoken language.
	How to correctly formalize the mapping of data, information
	and knowledge to physical stimuli
	and then to the perceived stimuli
	respecting the Weber-Fechner and Steven's laws at least for visual
	and auditory cues?
\end{abstract}

\section{Introduction}
Linked data is useful for data integration from different sources,
for semantic representation and manipulation of data,
such as by automated inferences,
and sound conceptual considerations of both domain and context
knowledge.
The mapping of data to audiovisual cues
involves a number of routines, conceptual frameworks,
and data.
The scaling and integration factors are decisive in setting
of Big Data such as in social data mining.
The study of data that expresses human social systems
poses ethical and conceptual challenges~\cite{an,an2}.
An overview of the Linked Open Social Data translated to
RDF and related ontologies and vocabularies
can be found in~\cite{nuvem1,tese,losd}.
Some limitations of current linked data standards
are described and potential solutions given in~\cite{ont}.

\section{Current Framework}\label{current}
We obtained compelling algebraic, empiric and conceptual
analyses of social structures~\cite{tese,stab},
often supported by software, which is a very efficient
way to represent
what is being performed (although also very incomplete).
In this section, we focus on current efforts around Audiovisual Analytics.

\subsection{AAVO}\label{sec:aavo}
AAVO is a first tentative ontology for Audiovisual Analytics.
Figure~\ref{aavo} depicts the AAVO Core,
and both Core and Extended AAVO are well described in~\cite{aavo}.
The following concepts are envisioned for the core
but not yet implemented:
Hypothesis, Analysis, Task/Purpose/Application.
Some relations seem less fundamental then others,
which might be expressed using the techniques described
in~\cite{ont}.
AAVO is very incipient, and will most probably receive
other layers of conceptualizations beyond Core and Extended,
such as for data type, colors, charts, etc.

\begin{figure}[htbp]
  \centering
  \includegraphics[width=15cm]{../figs/aavomin}
\caption{A graphical representation of current AAVO Core.
	More information in Section~\ref{sec:aavo}.}\label{aavo}
\end{figure}

Its purposes are: automated inferences and recommendations on
contexts involving data visualization (e.g. for suitable visualizations given a
dataset or vice-versa);
enable relevant theoretical discussions by having an
established and formal conceptualization;
provide a representation of data that is friendly for
both humans and machines (in browsing, discovery, inference);
underpin data integration;
provide the conceptual and data architecture for
the audiovisual analytics software described in
the next section.

\subsection{An audiovisual analytics platform for social systems}\label{sys}
A software system, a web platform for audiovisual analytics of big data,
has been envisioned with the following characteristics:
\begin{itemize}
  \item Interface has capabilities of making automated and periodic changes of
	  the presentation of the content (data and analyses).
		This entails a fractal and musical consideration
		of the audiovisualzations.
		User might pause and set changing patterns.
  \item The simultaneous use of both the vision and auditory channels.
	  User might mute or focus on the sonic cues.
  \item The system has the purpose of providing an aesthetic experience
	  to the user.
		This includes facilities to render audiovisual media
		for aesthetic appreciation or attractive representation of
		analyses~\cite{tufte}.
  \item Keeps records of state through users and sessions.
	  Each session has a set of states, each state a set
		of interface parametrization and data references.
	Annotations might be linked to users, sessions, states, data,
		analysis methods, widgets, etc; and categorized
		by keywords or simple values.
  \item Persistence (probably through the web browser's sandbox)
	  to allow a user to reload already downloaded software
		and data,
		and to allow a user to work offline.
  \item Sharing of sessions, states, data, and annotations among users,
	  preferably in real time (as achieved through Meteor.js).
		Sharing of media to non-users and for presenting results.
  \item A written interface to control the system which should be
	  suitable for scripting, one line commands, keyboard shortcuts.
		It will probably be JavaScript with Vim-like one-liners and scripting.
		(The written interface expresses a very broad space
		of possible commands.)
  \item Linked data to achieve a formal and browse-friendly representation of data for both humans and machines;
	  for a unified consideration of both data and knowledge architecture;
facilitates the linkage of metadata from users;
facilitates resources recommendations.
  \item Emphasis on the analysis of networks and text (and their simultaneous analysis),
	  reason why it is fit for social networks, but also for general complex networks
		and textual data.
  \item Designed to be enhanced with use by means of tests with users,
	  and of analyses of sessions and states
	  (maybe have facilities to record mouse movement, clicks and keyboard strokes).
\end{itemize}

The temporal dynamic of the interface is useful to do a rapid scan of the data by parts.
Observing data by parts is a core procedure in mining big data.
Persistence is also valued for the purpose of analyzing big data.
Complex networks and text yield a broad user base
and set of knowledge fields which might be studied withing the system.
The focus on social systems entails analysis results of interest
to scientific research, State and private sectors.
Art in current and envisioned developments
is useful for formal documents and favors the engagement
of layman.

\subsection{Perceptual framework}
Frequency and intensity are related to pitch and volume
through the Weber-Fechner law.
One might also think about durations on the same sense.
Visual estimuli, on the other hand, seem to be 
more often modeled as the Steven's law (is this true?).
Is it possible to relate frequency and intensity to
pitch and loudness through Steven's law?
Is the perception of visual cues also fit by Weber-Fechner?
The Steven's law is a power law while Weber-Fechner is an exponential
law:
\begin{equation}
	\Delta p = log_X(e_1/e_0)
\end{equation}
\noindent for Weber-Fechner and
\begin{equation}
	p = k.e_1^\alpha
\end{equation}
\noindent for Steven's,
where p is the (subjective) magnitude of the perceived stimuli,
$X$ and $\alpha$ depend on the stimuli,
and $e_0$ and $k$ depend on the the units used.
$\Delta p$ is an interval, a distance, in the perceived magnitude.

In equally spaced samples of the stimuli (e.g. in PCM audio),
the sample values might express amplitude or frequency.
If the sample values represent durations, the separation
between samples looses its meaning, but enables a Fourier
analysis in a 'virtual' or 'imaginary' space.
For example, consider the Fouriter transform of
this sequence of durations in seconds:
[1, 0.5, 0,5, 1, 0.5, 0.5].
The components are sinusoids whose amplitude
is duration.
The durations are the sample values,
each value corresponding to a sample separated
by a virtual distance.
% keep static amplitude or phase
% in virtual and real-simed Fourier transfoms

Related to sound, most traditional music notation and qualities
respect the Weber-Fechner law, with note grids (e.g. semitones, octaves),
respecting the logarithm of frequency: interval$=log_2(f_1/f_0)$,
loudness$=log_10(p_1/p_0)$.
Duration seems also to be conceptualized as such
because musical notation uses the division in two (simple, imperfectus),
amd three (perfectus) yielding a grid having
$d_i=d_{i+1}/2$ and $d_i=d_{i+1}/3$.
$durations=2^{x-i}$ and $=3^{x-i}$.
Which implies $log_2(dur) = x-i \Rightarrow i = log_2(X/dur), x = log_2X$.
A Weber-Fechner compliant relation
between the number of the durations and
depth of the division x-i = (4, 8, 16) for simple,
(3, 9, 12) for compound,
(5,7,13) complex.
6 is compound in traditional music theory,
but is the only one that relates periods of 2 and 3,

The Human Processor Model~\cite{HPM}
has somewhat established values for
a number of cognitive characteristics such as eye movement time (230ms),
half-life of visual image storage (200ms) and 
of
working memory (7s),
and visual and auditory capacities (17 letters, 5 letters).
These reference values, together with the
equations relating physical and perceived stimuli,
should be integrated to AAVO.

\subsection{Other: Erdös Sectioning, KS-derived statistics, Wordnet,
text mining, stability and differentiation, partnerships}
Other works, such as the classification of participants (vertices) as
hubs, intermediary and peripheral; statistical tests e.g. to
observe linguistic differences among the network participants;
the use of Wordnet and token-oriented methods for text mining;
and current partnership with researchers in physics, computer science,
mathematics, philosopy and social sciences, clinical and social psychology,
give some of the support for our endeavors.

\subsection{What to do?}
We will surely integrate the envisioned concepts into AAVO Core.
We should develop the system in Section~\ref{sys} to some extent,
maybe for concept proofs or simple prototyping.

With the research groups @Nuvem, 
especially those audiovisual and societal,
relate the missing concepts in AAVO,
enhance the software design of the system.
Know about the right tools for visualizing
ontologies and social data.
For knowing about pertinent analysis methods
and known conceptualizations.
might want to formalize their own conceptualizations,
and concepts related to AAVO might have similar developments
or which are confluent (e.g. symbiotic)
with Section~\ref{current}: software,
theoretical framework, research and institutional experiments and equipment,
linked data, and other initiatives or knowledge.








\begin{thebibliography}{99}

\bibitem{an}
	Fabbri, R. What are you and I? [anthropological physics fundamentals], 2015. Available at \url{https://www.academia.edu/10356773/What\_are\_you\_and\_I\_anthropological\_physics\_fundamentals\_}

\bibitem{an2}
	Anthropological physics and social psychology in the critical research of networks. Complex Networks Digital Campus (CS-DC'15).
	Em \\\url{http://cs-dc-15.org/papers/cognition/social-psychology/anthropological-physics-and-social-psychology-in-the-critical-research-of-networks/}

\bibitem{nuvem1} 
  R.~Fabbri and M. Pisani
  ``Analytic consideration of the society by itself,''
  I Workshop @NUVEM, UFABC campus de Santo André,
		Available at: \url{https://github.com/ttm/aavo/raw/master/latex/nuvem/nuvem1.pdf}.

\bibitem{tese}
Fabbri, R. (2017). Topological stability and textual differentiation in human interaction networks:
		statistical analysis, visualization and linked data. Doctoral thesis.
		From \url{https://github.com/ttm/thesis/raw/master/thesis-rfabbri.pdf}

\bibitem{losd}
	Fabbri, R., \& de Oliveira, O. N. (2016). Linked Open Social Database. Github repositories, from \url{https://github.com/ttm/linkedOpenSocialDatabase/raw/master/paper.pdf}

\bibitem{ont} 
  R.~Fabbri and M. Pisani,
  ``Enhancements of linked data expressiveness for ontologies,''
  Anais do XX ENMC - Encontro Nacional de Modelagem Computacional e
  VIII ECTM - Encontro de Ci\^encias e Tecnologia de Materiais, Nova Friburgo,
  RJ - 16 a 19 Outubro 2017
		Available at: \url{https://github.com/ttm/equanimousScalefree/raw/master/article/article.pdf}.

\bibitem{stab}
Fabbri, R., Fabbri, R., Antunes, D. C., Pisani, M. M., \& de Oliveira, O. N. (2017). Temporal stability in human interaction networks. Physica A: Statistical Mechanics and its Applications.

\bibitem{aavo} 
  R.~Fabbri,
		``Audiovisual Analytics Vocabulary and Ontology (AAVO): initial core and example expansion ,''
  Anais do XX ENMC - Encontro Nacional de Modelagem Computacional e
  VIII ECTM - Encontro de Ci\^encias e Tecnologia de Materiais, Nova Friburgo,
  RJ - 16 a 19 Outubro 2017
		Available at: \url{https://arxiv.org/abs/1710.09954}.

\bibitem{tufte}
	Tufte, E. R. (1990). Envisioning information. Graphics press.

\bibitem{HPM}
	Human processor model. (2017, March 11). In Wikipedia, The Free Encyclopedia. Retrieved 06:16, November 13, 2017, from \url{https://en.wikipedia.org/w/index.php?title=Human_processor_model&oldid=769706845}

\end{thebibliography}


\end{document}
