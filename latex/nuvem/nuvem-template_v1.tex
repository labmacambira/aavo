\documentclass[letterpaper,10pt]{article}
\usepackage{nuvem}
\usepackage[utf8]{inputenc}


\begin{document}

\title{Analytic consideration of the society by itself}

\author{Marilia M. Pisani$^1$\\
Renato Fabbri$^2$}
\address{$^1$CCNH/UFABC, $^2$IFSC/USP}
\email{$^1$marilia.m.pisani@gmail.com, $^2$renato.fabbri@gmail.com}

\begin{abstract}
Social participation had been growing on the last decade
and had a drawback with the recent rise of conservative political parties 
in various locations of the world.
We aim to present a set of developments on linked social data
and analysis over the last years, with a focus
on the potential use for the civil society and scientific academy.
Various conceptualizations have been gathered by interviewing specialists
and State authorities and have been validated by them and their institutions.
Data from known social networks, such as Facebook, Twitter, IRC and Email,
and from more specialized social participation platforms, have been
	translated to RDF and linked to these conceptualizations in the form of OWL ontologies~\cite{losd,pnud5}.
Social networks were reported as very stable and their language varies
	with connectivity~\cite{tese,stab}.
	Also, resources recommendation and experiments have been performed with such data.
	How are we to articulate the gathering and analysis 
	of such data with the needs of the civil society and the academy?
	Will it sufice to deal with the private and State interests
	that shape our society?
	Does Anthropological Physics yield reasoable strategies
	to collect and analyze data from our society?
	How does the global semantic web of linked data~\cite{GGG}
	relate to linked social data?
\end{abstract}

\section{Introduction}
Social analytics is a term associated to Lars-Hendrik Schmidt~\cite{cite}
in philosophy.
One might also relate social analytics to social data mining,
by extension of the terms: visual analytics, data analytics, social media analytics etc,
being used nowadays~\cite{wVA,wDA,wSMA,wSNA}.
Interestingly, both social analytics 
have among core practices the reporting of tendencies of the times.
And they put emphasis on issues and analytical outcomes
that are very controversial and result from recent technological and scientific developments.

On the critical theory tradition,
Marcuse have a central role on
questioning the conditions
imposed for living by technological advances~\cite{m1,m2,m3}.
And the consideration of the authoritarian personality traces~\cite{au}
have gained relevance due to unforeseen social phenomena
such as the recent and worldwide rise of the right-winged manifestations
and political representation~\cite{rise}.

On physics and computer sciences,
both the advent of the complex networks
theory and the big data practices
had an impact on the utility of considering
social systems.
One enables forecast of properties
and the use of natural laws for deriving understandings.
The other provides the data that represents the real systems
in scales that make them useful for management and
scientific studies (including experiments as in Section~\ref{exp}).
Social networking platforms are massively and constantly used and thus fed.
They are most often elaborate suites of software with
an interface for the user to interact with other users
and often makes use of these recent scientific and technological
developments.

In this context,
we advocate the analytic consideration
of the Society by itself.
There are many ways to achieve this,
and it is performed to some extent.
Our focus is on the mining of the data
derived from social platforms.
It is not very trivial in our context
of monetary obsession to achieve engagement
in such analytical undertake.
Even if there is money making in the process,
the citizens already have their livelihood activities.
Therefore, some effort has been given to
enable this kind of `social participation'.
Core goals are:
\begin{itemize}
	\item Develop the critical view, on specialists and layman,
		of social media and
		management performed by State and private sector.
	\item Make advances in the data available for consideration
		and the algorithms for analysis.
	\item Implement or study the possibility implementation
		of software systems.
\end{itemize}

\section{What has been done}
On the following sections, we will focus on
developments related to social systems and technology.
There are methods, software and scientific discoveries
that underpins them~\cite{stab,tese,kolm}.

\subsection{Social participation data}
Data from Facebook, Twitter, Email lists, IRC Channels, ParticipaBR,
Cidade Democrática and AA have been translated into linked data (RDF)
and has been named the LOD (Linked Open Social Data~\cite{lod}).
Initial OWL ontologies were derived from the data by automated routines.
The dataset is oriented towards scientific research
and the development of analytic interfaces~\cite{p1,p2,p3,p4,p5}.
Even so, it has been used by participants of a SESC oficina~\cite{sesc},
and has received some attention from the scientific community~\cite{nexos,tese}.

\subsection{Ontologies}
Beyond the ontologies mentioned in the last section,
there are the social participation ontologies that describe
social participation instances and mechanisms~\cite{p5}.
They are: 
These ontologies are contextualized by the whole semantic web
(or Giant Global Graph~\cite{ggg})
and by other social participation ontologies:
the OPa (Ontology of ParticipaBR~\cite{p5}), the OPS (Ontology of Social Participation~\cite{ops}),
the OCD (Ontology of Cidade Democrática~\cite{p5}), the OAA (Ontology of AA~\cite{oaa}).
There are minor ontologies, such as the one about the Caixa Mágica~\cite{cm}.
These OWL ontologies (and SKOS vocabularies that most of them have) enable
sound conceptual discussions, navigation by humans and machine (data discovery),
 automated inference, and data organization, integration and linkage by means of conceptualizations.
It seems reasonable to use the ontologies to link the social participation data
and enable a scalable participatory legacy of machine- and human- friendly representation
of data for navigation, query and analysis, automated inference
and conceptual considerations.

\subsection{Critical theory and anthropological physics}
The potential for prejudice and the study of the self
were considered in~\cite{imp,stab,tese}.
In summary, the hub, intermediary and peripheral sectorialization
of a network might be done in a social network.
It yields a classification of the participants
by the comparison of the degree (or strength) distribution
against that of an Erdös-Rényi network.
Hubs are valued by literature
(although the intermediary are also reported as authorities
and structuraly important agents)~\cite{ega},
which favors the potential for prejudice.
At the same time, the roles participant


The cold-blooded reception of the reality
enabled technologial devices such as drones~\cite{drones}.

<últimos trabalhos marilia, e.g. impulso>


The anthropological aspect of observing natural
phenomena in the social structures
of the observer him/herself has been
considered in~\cite{an,an2,imp,tese}.
Comte talked about anthropological physics,
but it had the sense which physical (or biological or natural)
anthropology has, which is a main field of anthropology
(the other is social anthropology).
Anthropological physics, in the sense we use and understand correct,
is physics (research on the natural phenomena of complex networks)
with anthropological aspects.
The writing and study of diaries is a traditional ethnographic asset,
which resembles the collection and study of one's data by the person itself,
social data.
The current guidelines are centered on keeping the process open,
using free and open source tools and file formats,
and publicly accessible repositories of software, gadgets, data, etc;
while gathering and studying the social data related to the researcher
(and other data if needed and afterwards).



\subsection{Self-transparency, AA and the fundamental cycle}
A system for sharing work, development and research processes
(actually, any dedication)
is described in~\cite{aa,aa2}.
It is the Algorithmic Autoregulation (AA) methodology and it has
received a number of software implementations.
It is based on voluntary logging of messages (called shouts)
about what is being done (reading, coding, etc) that might
be linked to sessions and screencasts or be blind-reviewed
by other users.
The methodology and software have initiated discussions
e.g. about documenting academic dedication not only though
documents (e.g. articles, books) and participation (e.g. conferences, thesis defense),
which would enable one to dedicate more time to individual
tasks and make more paced and relevant contributions.

Also, the conception of the 


\subsection{Audiovisual Analytics platform}
for social data, with focus on network and textual
		data.                

\section{What shall we do?}
Most certainly we will
make the LOSD and social participation ontologies (Sections~\ref{losd} and~\cite{ont})
available e.g. on DataHub and Data.World.
In the Nexos research network, we
considered a number of times to study authoritarian personality traces~\cite{au}
on this data.
And to further consider the Anthropological Physics aspects that
arise from studying our own social structures~\cite{af,af2}.
The self-transparency is having some attention in scientific writings
and hacks~\cite{aa,aa2} but seems to be lacking user bases.
The developments for Audiovisual Analytics are incipient
and has the development focus of only one researcher and developer.
The Nexos research network is very active nation wide.
It has an overall tendency to tackle subjects within the
critical theory tradition and thus considers the society
and potential enhancements through criticism of its organization.

Research groups @Nuvem of Cloud Computing and Intelligent Societies have 
further context to share about these issues.
For example, are there SparQL endpoints with social participation
(including self-transparency) data?
Is it necessary to contact lawyers to
better know the limits of our possibilities
to gather and research our social data within
the Anthropological Physics perspective
and are there very well known guidelines?
Is there more data participatory linked data available
in Brazil? Ontologies?
We might use directions 
on better linking LOSD to the semantic web
(DBPedia, other participatory data, etc),
and for a reasonable way to keep the data online
(through DataHub or Data.World or both?
A Pubby-like interface?),
and to develop an audiovisual analytics
software
(persistence, web?, analysis methods, audiovisual rendering).

How to enable a self-transparency user base?
Where to keep the linked data?
How to manage the ontologies and keep their
constant development (as needed and predicted in the literature)?
How to achieve a reasonable posture
and use of our social (open linked) data?
Is it possible to have an audiovisual analytics
platform with our own (scraped) social data,
with facilities for media rendering,
interaction experiments (collection and diffusion of information),
that enables the user base and interested parties to
collective gather social data, analysis and conceptualizations?
Is the societal consideration of our social data
relevant for equilibrium with the State and private sectors?





% keep the data online
% develop an audiovisual analytics software
% maintain partnerships between humanities and hard sciences

    * Como podemos trabalhar o tema:
    conexão com Nuvem, com Nexos e com IFSC/ICMC.

\section{Conclusions}

\begin{thebibliography}{99}

\bibitem{krishnan00} E. Krishnan, A. M. Shan, T. Rishi, L. A. Ajith, C. V.
Radhakrishnan, \textit{On-line Tutorial on \LaTeX{}},
``Mathematics'' (Indian \TeX{} Users Group, 2000), \\
\url{http://www.tug.org/tutorials/tugindia/chap11-scr.pdf}.

\bibitem{vantrigt97} C. van Trigt, ``Visual system-response functions and estimating reflectance,''
J. Opt. Soc. Am. A \textbf{14}, 741--755 (1997).

\bibitem{masters93} T. Masters, \emph{Practical Neural Network Recipes in C++} (Academic, 1993).

\bibitem{shoop97} B. L. Shoop, A. H. Sayles, and D. M. Litynski, ``New devices for optoelectronics: smart pixels,''
in \emph{Handbook of Fiber Optic Data Communications},
C. DeCusatis, D. Clement, E. Maass, and R. Lasky, eds. (Academic, 1997), pp. 705--758.

\bibitem{kalman76} R. E. Kalman,``Algebraic aspects of the generalized inverse of a rectangular matrix,'' in
\emph{Proceedings of Advanced Seminar on Generalized Inverse and Applications}, M. Z. Nashed, ed. (Academic, 1976), pp. 111--124.

\bibitem{craig96} R. Craig and B. Gignac, ``High-power 980-nm pump lasers,''
in \emph{Optical Fiber Communication Conference}, Vol. 2 of 1996 OSA Technical Digest Series (Optical Society of America, 1996), paper ThG1.

\bibitem{steup96} D. Steup and J. Weinzierl, ``Resonant THz-meshes,''
presented at the Fourth International Workshop on THz Electronics, Erlangen-Tennenlohe, Germany, 5--6 Sept. 1996.

\end{thebibliography}


\end{document}
